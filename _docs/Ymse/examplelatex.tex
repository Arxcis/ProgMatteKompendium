\documentclass[12pt]{amsart}
\usepackage{amsfonts}
\usepackage{amssymb}

\title{Examplefile in latex}
\author{Bernt Jensen}

\newtheorem{theorem}{Theorem}
\newtheorem{lemma}[theorem]{Lemma}
\newtheorem{definition}[theorem]{Definition}
\newtheorem{example}[theorem]{Example}
\newtheorem{exercise}{Exercise.}


\begin{document} 

\maketitle

\section{Introdution}

\subsection{Here we go}



You write text as normal. Math needs to be written within dollar signs like this $4+5=19$ or $$50+30=100$$ to put it on a separate line.
Fractions are written as $\frac{13}{12}$. Functions can be written like $sin(x)$ or $\mathrm{sin}(x)$ depending on what you like.
Here is a matrix $\begin{pmatrix}1 & 2 & 3 \\ 4 & 5 & 6 \\ 7 & 8 & 9\end{pmatrix}$. Some logical symbols are $\wedge$, $\vee$, $\oplus$, $\rightarrow$,
$\leftarrow$ and $\leftrightarrow$. Greek symbols $\alpha$, $\beta$. You can also use equations. Per cent is $\%$ and $\times$ is used for vector product.

\begin{equation} \label{eq1}
4\cdot 4 \cdot 5
\end{equation}

which you can refer\footnote{this is a footnote} to if you add a label. This refers to the equation above \ref{eq1}.

you can write text within mathmode by $$4+3=7 \mbox{ this is some text within dollars}$$
$$1,2,3,\cdots,10$$
If you want to quote code you should write them in verbatim

\vspace{10pt}

\begin{verbatim}
for(int i = 0; i ++; i < 100) {
      foo();  // but that tab doesn't work,
                // at least on my computer
}
\end{verbatim}

\subsection{Here we go again}

\section{And one more comment}


When you write text with mathematics you never never never start with a mathematical symbol or a variable,






\end{document}




